%
% File yantosca_COSC6336_assg_1.tex
%
%% Based on the style files for NAACL-HLT 2018, which were
%% Based on the style files for ACL-2015, with some improvements
%%  taken from the NAACL-2016 style
%% Based on the style files for ACL-2014, which were, in turn,
%% based on ACL-2013, ACL-2012, ACL-2011, ACL-2010, ACL-IJCNLP-2009,
%% EACL-2009, IJCNLP-2008...
%% Based on the style files for EACL 2006 by
%%e.agirre@ehu.es or Sergi.Balari@uab.es
%% and that of ACL 08 by Joakim Nivre and Noah Smith

\documentclass[11pt,a4paper]{article}
\usepackage[hyperref]{naaclhlt2018}
\usepackage{times}
\usepackage{latexsym}
\usepackage{url}
\usepackage{alltt}
\usepackage{amsmath}
\usepackage{mathtools}
\usepackage{amssymb}
\usepackage{pgfplots}
\usepackage{pgfplotstable}
\pgfplotsset{
  tick label style={font=\tiny},
  label style={font=\footnotesize},
  legend style={font=\tiny},
  compat=newest,
  empty line=scanline
}
\pgfplotstableset{
  col sep=comma,
}

\pgfplotstableread{./pmicro.csv}\pmicro
\pgfplotstableread{./rmicro.csv}\rmicro
\pgfplotstableread{./amicro.csv}\amicro
\pgfplotstableread{./f1micro.csv}\fmicro
\pgfplotstableread{./pmacro.csv}\pmacro
\pgfplotstableread{./rmacro.csv}\rmacro
\pgfplotstableread{./amacro.csv}\amacro
\pgfplotstableread{./f1macro.csv}\fmacro

\aclfinalcopy % Uncomment this line for the final submission
%\def\aclpaperid{***} %  Enter the acl Paper ID here

%\setlength\titlebox{5cm}
% You can expand the titlebox if you need extra space
% to show all the authors. Please do not make the titlebox
% smaller than 5cm (the original size); we will check this
% in the camera-ready version and ask you to change it back.

\newcommand\BibTeX{B{\sc ib}\TeX}

\title{HMM-HPS: A Heuristic Preseed Approach to Part-of-Speech Tagging with Hidden Markov Models}

\author{Michael Yantosca \\
  Department of Computer Science, University of Houston / Houston, TX \\
  {\tt mike@archivarius.net} \\}

\date{2018-03-04}

\begin{document}
\maketitle
\begin{abstract}
  Despite their simplicity, Hidden Markov Models can often
  provide surprisingly accurate predictions for tasks such
  as part-of-speech tagging. By adding the results of
  heuristic functions against given observations to the
  observed feature vector and preseeding the probability
  matrices of the model accordingly, the accuracy of the
  model can be substantially improved even training
  against small corpora.
\end{abstract}

\section{Introduction}

\paragraph{}
Part-of-speech (POS) tagging, i.e., the annotation of words in linguistic corpora with syntactic
markers, serves a crucial role in the natural language processing pipeline as it provides
a syntactic skeleton without which proper semantic analysis downstream would prove extremely difficult.
However, the process of tagging is labor intensive, and as corpora have grown, the desire
for expediting the process has driven development of automated POS taggers. Early implementations
employed rule-based methods, but such approaches tended not to scale in the general case.

\paragraph{}
Alternative approaches using Hidden Markov Models (HMMs) to model a generative probability
distribution whereby state transitions between the different parts of speech emit lexical
observations have proven reasonably accurate.
To explore the exact nature of the efficacy of HMMs in POS tagging and what further
augmentations might prove beneficial, a HMM model was built from scratch, trained,
tested, and refined in the \emph{COSC 6336 Part-of-Speech Tagging with HMM} competition
on \texttt{competitions.codalab.org} under the user handle \texttt{ghostant}.

\section{Methodology}

\paragraph{Training}
Initial development of the POS tagger began with building a frequency counter for
arbitrary order n-grams encountered over the course of reading the training file.
These n-grams included the $n$ pairs of words and corresponding language identification
tags up to the current word being read. As each new word was encountered, each
order of n-gram leading up to the word was counted. Preceding words and language ID
tags were kept in a bounded cursor to minimize processing overhead. If the cursor
was full when it encountered a new word, the oldest entry was excised so that the
new entry could be introduced and tallied accordingly.

\paragraph{}
Additionally, the count of each annotated POS tag was recorded, along with the count
of the transition from the previously seen POS tag. At the beginning of a sentence,
an implicit \texttt{Q0} tag served as the origin of the transition, and at the end
of a sentence, an implicit \texttt{QF} tag served as the terminus of the transition.
The explicit tag definitions in the training corpus employed the Universal Dependencies
tagset \cite{UPOS}.

\paragraph{}
Once the training file had been read and the frequency model established,
the frequency counts were converted to probabilities. At first, Laplace smoothing
\cite[47]{JurafskyMartin}
with equiprobable weights was used to supply probability mass for out-of-vocabulary
words and unseen state transitions, but this was ultimately generalized to add-$k$
smoothing \cite[49]{JurafskyMartin} to permit tuning based on empirical results.
Other smoothing methods were briefly considered but rejected on account of the time
involved to implement them given the schedule of the competition.

\paragraph{}
No attempt was made to store the model in a form that could be loaded later,
so training was done online in the same execution run as the testing.
Whereas this significantly slowed development velocity at the beginning,
the addition of later refinements and the stream-processing approaches
employed in both training and testing made this an acceptable cost.

\paragraph{}
It should be noted that all training was done with the provided \texttt{train.conll}.
On account of time constraints, more sophisticated methods of cross-validation \cite[86-87]{JurafskyMartin}
were not employed.

\paragraph{Testing}
The testing phase employed the Viterbi decoding algorithm as given in Jurafsky and
Martin \shortcite[131-134]{JurafskyMartin}. To save on computation time and memory usage,
a running $a_{max}$ value was kept in the innermost loop for determining each
state's most likely predecessor.

\paragraph{}
To gauge efficacy of the tagger during development and enable rapid prototyping,
a rough measure of precision was taken via the following shell command:

\begin{alltt}
  \tiny{diff dev.conll <output file> | egrep '<' | wc -l}
\end{alltt}

\paragraph{}
All development testing prior to submission was done with the provided
\texttt{dev.conll}, and this drove decisions on refining and tuning the model.
Submissions to the competition were only trained with \texttt{train.conll}.

\paragraph{Refinements}
After the first pass of the tagger pipeline was completed, it was clear that it would
not be able to compete effectively on its own. Rough estimates of precision from
initial assays against \texttt{dev.conll} ran around 0.81, which was below the
Naive Bayes competition submission made by the user blackbishop.
Incorporating some of the ideas expressed by Dr. Solorio during lecture on February, 28, 2018,
a heuristic fallback step was introduced in the decoding phase for unknown vocabulary.

\paragraph{}
The heuristic fallback employed regular expressions to capture whether the unigram
in the emission was capitalized, contained only punctuation marks, or only alphanumeric
characters. In the capitalization case, the fallback emission probability was zeroed out
for every class except \texttt{PROPN}. In the case where only punctuation marks were
present, the fallback emission probability was zeroed out for every class except
\texttt{PUNCT}. In the case where only alphanumeric characters were present, the fallback
emission probabilities for $P(word|\texttt{PUNCT})$ and $P(word|\texttt{SYM})$ were
zeroed out.

\paragraph{}
This simple heuristic check was done as it was noticed that many clearly nominal or
verbal tokens were being labeled as \texttt{PUNCT}. The refinement performed as expected,
raising the correctness of the output, if not perfectly preserving the probability masses
involved. Adding this refinement also incurred a massive performance penalty since
the series of branching if-statements became quadratic with respect to $Q$, the set
of possible generating POS states.

\paragraph{}
After the addition of heuristic fallback, attention was turned to combining the
emission probabilities of different orders of n-grams. Two modes of operation were
implemented to achieve this: backoff and interpolation \cite[49-50]{JurafskyMartin}.
The backoff mode simply backed off to increasingly lower orders of n-grams from the
highest order stored by the model until it reached the unigram order.
If the model had not seen the token, it would fall back to a special unknown token
encoded as \texttt{<UNK>} that had been seeded with smoothed probabilities if one
of the heuristic fallback cases did not apply.

\paragraph{}
The interpolation mode was implemented in a rudimentary fashion. The highest order n-gram
would receive a weight of 0.5, the (n-1)-gram would receive half of the remaining weight,
and so on until the unigram observation. If any of the n-grams had not been seen, its
contribution would fall back to the heuristic fallback unigram of \texttt{<UNK>}.
At the beginnings of sentences and in sentences shorter than the highest order n-gram
in the model, no phantom n-grams were computed. The interpolation chain would start with
the longest observed n-gram and apply the formula in the same manner.

\paragraph{}
An issue was encountered in the dev set with a sentence composed of 91 tokens. A numerical
underflow in the multiplication of probabilities during the decoding phase led to a state
where the backpointers could not be traversed since they would only be assigned if a
preceding state had greater than zero probability, the initial value of the $a_{max}$
cursor. To counter this, a major refactoring of the code was done to store and use
the logarithm of the various probabilities calculated and multiplication operations
were changed to addition. To provide the same effect in log space, the value zero
was replaced with $-\texttt{sys.float\_info.max}$.

\paragraph{}
Since the abysmal execution speed  of the heuristic fallback put a drag on development time,
the heuristic intuitions were integrated earlier in the pipeline by preseeding the
emission probabilities that corresponded to the various heuristic checks. Three
heuristic check responses were encoded in each observation key, and the regular
expression match that provided the key was calculated once upon ingestion during
the decoding phase. The three features that were added to the word and language ID
as part of the observation were as follows:

\itemize
\item{\texttt{SOME\_WORD}: at least one alphanumeric character}
\item{\texttt{CAP}: capitalized alphanumeric followed by alphanumeric}
\item{\texttt{ALL\_PUNCT}: only punctuation characters }

\paragraph{}
This provided a significant performance boost that enabled development to progress
more swiftly while maintaining the same level of precision. Some attempts were made to
redistribute the probability mass appropriately, but in most cases heuristic downgrades
and upgrades during the conversion of frequencies to probabilities were done unilaterally
for the sake of time.

\paragraph{}
Furthermore, the iterative decay of significance in the weights for the rough
interpolation of multiple orders of n-grams was modified to give more credence
to higher-order n-grams. Instead of $\frac{1}{2}$ of the remaining weight, each n-gram
in the chain received $\frac{3}{4}$ of the remaining weight.

\paragraph{}
As a final step, emission probabilities for POS states \texttt{INTJ}, \texttt{UNK},
and \texttt{X} were downgraded for unknown words that matched the \texttt{SOME\_WORD}
feature. This was done as a result of observations that these tags seemed to be
disproportionately assigned in pre-submission test runs on \texttt{dev.conll}.
The decision was made in the full understanding of the risk of overfitting on the
presumption that unknown ``dictionary'' words would be far more likely to turn up
in a corpus than novel interjections owing to the former's richer semantic content.
This intuition is borne out when taking a cursory glance at other corpora \cite{WFD100K}
used in POS tagging research \cite{CLAWS7} \cite{BrysbaertNewKeuleers}. For example,
sampling every twentieth word from the list of 100,000 words based on the 450 million word
Corpus of Contemporary American English (COCA) yields a mere four interjections in the set of
5,000, the most frequent being ``goodbye'' with a frequency ranking of 9,200 out of 100,000 \cite{COCA5K}.

\section{Experimental Results}

\subsection{Competition Submissions}
Results from submissions to the competition from training on \texttt{train.conll}
and testing on \texttt{test.conll} were as follows\footnote{The first submission used regular probabilities. Log-space probabilities were used for all other submissions.}:

\paragraph{}
{\small
\begin{tabular}{llll}
  Accuracy & $n$ & $k$ & Heuristic \\
  0.9181 & 1 & 1 & fallback \\
  0.9352 & 1 & 0.01 & fallback \\
  0.9347 & 2 & 0.05 & preseed \\
  0.9349 & 1 & 0.01 & preseed \\
  0.9355 & 4 & 0.01 & preseed, revised weights
\end{tabular}}

\subsection{Local Experimentation}
While the competition submissions exhibited decent overall accuracy,
a more thorough examination of the model's performance in
terms of accuracy, precision, recall, and $F_{1}$-measure \cite[83-84]{JurafskyMartin}
controlling for certain parameters would provide a more incisive inquiry
into the strengths and weaknesses of this hybrid heuristic HMM approach.
In the interest of ensuring clarity in the derivation of these measures,
the following values are defined on a per-class basis:

{\tiny
\begin{align*}{}
  WC =&\ \texttt{\$(egrep -v '\^\$' dev.conll | wc -l)} \\
  GP =&\ \texttt{\$(egrep '\emph{class}\$' dataset/dev.conll | wc -l)} \\
  GN =&\ WC - GP \\
  FN =&\ \texttt{\$(diff dev.conll ../hw1-results/dev4.txt | } \\
     &\  \texttt{egrep '<' | egrep '\emph{class}\$' | wc -l)} \\
  FP =&\ \texttt{\$(diff dev.conll ../hw1-results/dev4.txt | } \\
     &\  \texttt{egrep '>' | egrep '\emph{class}\$' | wc -l)} \\
  TP =&\ GP - FP \\
  TN =&\ GN - FN
\end{align*}}

\paragraph{}
The per-class statistics were composited using both micro-averaging
and macro-averaging. The composite results are given in the
following graphs and tables.

\paragraph{Precision}
The micro-averaged precision matches very nearly the rough performance
characteristic that drove development. Precision decreased as
the constant used for add-$k$ smoothing increased, which is to be
expected given that the probabilities for heretofore unseen
observations were equiprobable except in cases where they were
overridden by heuristic measures. The performance of the system
lags well behind even simple naive Bayes systems when heuristic
methods are not employed, but using the weighted interpolation
with a small $k$ constant seems to minimize the occurrence of
false positives.

\begin{tikzpicture}
  \begin{axis}
    [ ybar,
      bar width=0.0625,
      legend style={at={(0.5,-0.15)},
        anchor=north,legend columns=-1},
      xtick=data,
      point meta=explicit symbolic,
      title=Precision (Micro-averaged)
    ]
    \addplot table [x=n,y=bfb,meta=k] {\pmicro};
    \addlegendentry{bfb};
    \addplot table [x=n,y=bps,meta=k] {\pmicro};
    \addlegendentry{bps};
    \addplot table [x=n,y=bnh,meta=k] {\pmicro};
    \addlegendentry{bnh};
    \addplot table [x=n,y=wfb,meta=k] {\pmicro};
    \addlegendentry{wfb};
    \addplot table [x=n,y=wps,meta=k] {\pmicro};
    \addlegendentry{wps};
    \addplot table [x=n,y=wnh,meta=k] {\pmicro};
    \addlegendentry{wnh};
  \end{axis}
\end{tikzpicture}

{\tiny\pgfplotstabletypeset[header=true]{\pmicro}}

Conversely, the macro-averaged precision values are significantly lower.
The holistic precision of the system may be well-suited to the task
and particularly the training and dev sets, but these results suggest
that a few predominant classes in which the system performs well
are masking the poor performance in the case of less frequently occurring
POS tags.

\begin{tikzpicture}
  \begin{axis}
    [ ybar,
      bar width=0.0625,
      legend style={at={(0.5,-0.15)},
        anchor=north,legend columns=-1},
      xtick=data,
      point meta=explicit symbolic,
      title=Precision (Macro-averaged)
    ]
    \addplot table [x=n,y=bfb,meta=k] {\pmacro};
    \addlegendentry{bfb};
    \addplot table [x=n,y=bps,meta=k] {\pmacro};
    \addlegendentry{bps};
    \addplot table [x=n,y=bnh,meta=k] {\pmacro};
    \addlegendentry{bnh};
    \addplot table [x=n,y=wfb,meta=k] {\pmacro};
    \addlegendentry{wfb};
    \addplot table [x=n,y=wps,meta=k] {\pmacro};
    \addlegendentry{wps};
    \addplot table [x=n,y=wnh,meta=k] {\pmacro};
    \addlegendentry{wnh};
  \end{axis}
\end{tikzpicture}

{\tiny\pgfplotstabletypeset[header=true]{\pmacro}}

\paragraph{Recall}
The results for micro-averaged recall approximate the
results observed for micro-averaged precision. Taken as a
whole, the system does not seem to be excessively prone to
false negatives.

\begin{tikzpicture}
  \begin{axis}
    [ ybar,
      bar width=0.0625,
      legend style={at={(0.5,-0.15)},
        anchor=north,legend columns=-1},
      xtick=data,
      point meta=explicit symbolic,
      title=Recall (Micro-averaged)
    ]
    \addplot table [x=n,y=bfb,meta=k] {\rmicro};
    \addlegendentry{bfb};
    \addplot table [x=n,y=bps,meta=k] {\rmicro};
    \addlegendentry{bps};
    \addplot table [x=n,y=bnh,meta=k] {\rmicro};
    \addlegendentry{bnh};
    \addplot table [x=n,y=wfb,meta=k] {\rmicro};
    \addlegendentry{wfb};
    \addplot table [x=n,y=wps,meta=k] {\rmicro};
    \addlegendentry{wps};
    \addplot table [x=n,y=wnh,meta=k] {\rmicro};
    \addlegendentry{wnh};
  \end{axis}
\end{tikzpicture}

{\tiny\pgfplotstabletypeset[header=true]{\rmicro}}

Once again, however, the macro-averaged recall indicates
that there are certain underrepresented classes where the
occurrence of false negatives is high.

\begin{tikzpicture}
  \begin{axis}
    [ ybar,
      bar width=0.0625,
      legend style={at={(0.5,-0.15)},
        anchor=north,legend columns=-1},
      xtick=data,
      point meta=explicit symbolic,
      title=Recall (Macro-averaged)
    ]
    \addplot table [x=n,y=bfb,meta=k] {\rmacro};
    \addlegendentry{bfb};
    \addplot table [x=n,y=bps,meta=k] {\rmacro};
    \addlegendentry{bps};
    \addplot table [x=n,y=bnh,meta=k] {\rmacro};
    \addlegendentry{bnh};
    \addplot table [x=n,y=wfb,meta=k] {\rmacro};
    \addlegendentry{wfb};
    \addplot table [x=n,y=wps,meta=k] {\rmacro};
    \addlegendentry{wps};
    \addplot table [x=n,y=wnh,meta=k] {\rmacro};
    \addlegendentry{wnh};
  \end{axis}
\end{tikzpicture}

{\tiny\pgfplotstabletypeset[header=true]{\rmacro}}

\paragraph{Accuracy}
No distinction is made between micro-averaging and macro-averaging
of accuracy since the results are mathematically the same.

\begin{tikzpicture}
  \begin{axis}
    [ ybar,
      bar width=0.0625,
      legend style={at={(0.5,-0.15)},
        anchor=north,legend columns=-1},
      xtick=data,
      point meta=explicit symbolic,
      title=Accuracy
    ]
    \addplot table [x=n,y=bfb,meta=k] {\amicro};
    \addlegendentry{bfb};
    \addplot table [x=n,y=bps,meta=k] {\amicro};
    \addlegendentry{bps};
    \addplot table [x=n,y=bnh,meta=k] {\amicro};
    \addlegendentry{bnh};
    \addplot table [x=n,y=wfb,meta=k] {\amicro};
    \addlegendentry{wfb};
    \addplot table [x=n,y=wps,meta=k] {\amicro};
    \addlegendentry{wps};
    \addplot table [x=n,y=wnh,meta=k] {\amicro};
    \addlegendentry{wnh};
  \end{axis}
\end{tikzpicture}

{\tiny\pgfplotstabletypeset[header=true]{\amicro}}

The results observed here appear superficially impressive,
but the abnormally high values point to an overfitting in the model.
This becomes immediately apparent when one compares the individual
values of the accuracy in the different classes. For instance,
the \texttt{PUNCT} class boasts of accuracy measures as high
as 0.9997 with heuristic provisions, but only 0.8782 without
fallback or preseeding. And yet, the overall accuracy even without
heuristic support is well above the 97th percentile in all cases.
The empirical results here corroborate Jurafsky and Martin \shortcite[83-84]{JurafskyMartin}
in decrying accuracy as a useful statistic.

\paragraph{$F_{1}$-Measure}

Since $F$-measure is a function of precision and recall, the results are
similar for those observed for precision and recall but are included
here for completeness. It succinctly captures in one measure the
inherent weaknesses of the model.

\begin{tikzpicture}
  \begin{axis}
    [ ybar,
      bar width=0.0625,
      legend style={at={(0.5,-0.15)},
        anchor=north,legend columns=-1},
      xtick=data,
      point meta=explicit symbolic,
      title=F1 Measure (Micro-averaged)
    ]
    \addplot table [x=n,y=bfb,meta=k] {\fmicro};
    \addlegendentry{bfb};
    \addplot table [x=n,y=bps,meta=k] {\fmicro};
    \addlegendentry{bps};
    \addplot table [x=n,y=bnh,meta=k] {\fmicro};
    \addlegendentry{bnh};
    \addplot table [x=n,y=wfb,meta=k] {\fmicro};
    \addlegendentry{wfb};
    \addplot table [x=n,y=wps,meta=k] {\fmicro};
    \addlegendentry{wps};
    \addplot table [x=n,y=wnh,meta=k] {\fmicro};
    \addlegendentry{wnh};
  \end{axis}
\end{tikzpicture}

{\tiny\pgfplotstabletypeset[header=true]{\fmicro}}

\begin{tikzpicture}
  \begin{axis}
    [ ybar,
      bar width=0.0625,
      legend style={at={(0.5,-0.15)},
        anchor=north,legend columns=-1},
      xtick=data,
      point meta=explicit symbolic,
      title=F1 Measure (Macro-averaged)
    ]
    \addplot table [x=n,y=bfb,meta=k] {\fmacro};
    \addlegendentry{bfb};
    \addplot table [x=n,y=bps,meta=k] {\fmacro};
    \addlegendentry{bps};
    \addplot table [x=n,y=bnh,meta=k] {\fmacro};
    \addlegendentry{bnh};
    \addplot table [x=n,y=wfb,meta=k] {\fmacro};
    \addlegendentry{wfb};
    \addplot table [x=n,y=wps,meta=k] {\fmacro};
    \addlegendentry{wps};
    \addplot table [x=n,y=wnh,meta=k] {\fmacro};
    \addlegendentry{wnh};
  \end{axis}
\end{tikzpicture}

{\tiny\pgfplotstabletypeset[header=true]{\fmacro}}

\section{Conclusions}

While the system performed relatively favorably in the competition,
the faults in the system strongly recommend a refactoring of certain
key components. The small training set size commended itself to the
use of heuristics, but better performance on a purely probabilistic
basis might have been achieved with a more comprehensive set of
training data. A more rigorous examination of the proper redistribution
of probability mass is required at a minimum.

\paragraph{}
Until the final refinements, the system was most precise with an
observation order of 1, i.e., with unigrams. The original expectation
was that increasing the observation order would increase the precision
of the system, but the sparsity of the higher-order n-grams may have
rendered increasing the order parameter of the model useless except
for all but the most common idioms.

\paragraph{}
It was observed that decreasing the $k$ smoothing constant increased
precision in the scenarios tested during development, but the strategy
may not apply in cases with a large unknown vocabulary. The equiprobable
distribution engendered by naive add-k smoothing does not reflect the actual
distribution of POS tags in a given language and necessitated the addition
of heuristic support. A more sophisticated approach to smoothing using a held-out
corpus to determine a more realistic distribution for unknown vocabulary
would likely yield better results with less semi-manual intervention.
By the same token, interpolation could have been improved by determining weights
from an empirically derived and cross-validated battery of tests against a collection
of held-out corpora rather than the simple iterative remainder method that was employed.

\paragraph{}
Furthermore, future efforts would likely do well to attempt a morphological deconstruction
of observations during training since lexemes are not necessarily atomic. Key components
of function are encoded in all manners of inflection which could be leveraged to provide
more granular feature determination.

\paragraph{}
For all of the issues with the implementation as presented, there is an even more
fundamental weakness in the HMM approach in general. The Markov assumption that
the probability of a whole sequence can be estimated by immediate predecessors
fails to account for situations in which subsequent context determines the syntactic
relationship of words heretofore, such as in languages that employ postpositions,
or situations in which vital antecedent context precedes the current state by more
than what may be economical to store, such as in topic-prominent languages.

\paragraph{}
To this end, it may be worthwhile to investigate strategies that employ lookahead tactics
like skip-grams or to transfer the encoding of language ID into the state transition matrix
rather than the observation matrix to more readily account for cases in which the syntax of
the two or more languages in a code-switched corpus follow a different transitional graph,
such as the adjective-noun noun-adjective dichotomy between English and Spanish.
Enriching the POS transition graph may even yield some insights, at least on a syntactic
level into the nature of code-switching itself.

\section*{Acknowledgments}

The author wishes to acknowledge Dr. Thamar Solorio for heuristic feature suggestions in
lecture on February 28, 2018; the NAACL template authors for providing a ready-made
{\LaTeX} template; and Dr. Daniel Jurafsky and Dr. James H. Martin for succinctly laying out the
properties and implementation of hidden Markov Models in the 3rd edition draft of their
text \emph{Speech and Language Processing}. The author further wishes to acknowledge
Dorothea Yantosca for her indispensable design criticism on the graphs and tables
which appear in this paper.

\bibliography{yantosca_COSC6336_assg_1}
\bibliographystyle{acl_natbib}

\appendix

\section{Supplemental Material}
\label{sec:supplemental}

\end{document}
